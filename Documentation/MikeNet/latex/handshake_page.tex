This page explains exactly what happens when a client attempts to connect to the server, and the impact that disabling the TCP handshake will have.\par
\par
\hypertarget{handshake_page_handshakeProcess}{}\section{The Process}\label{handshake_page_handshakeProcess}
The below process is implemented by \hyperlink{class_net_instance_client}{NetInstanceClient} and \hyperlink{class_net_instance_server}{NetInstanceServer}. \par
\par


size\_\-t is either a 32 bit unsigned integer or a 64 bit unsigned integer depending on which build of the API you are using. If you are running the 32 bit version, 4 bytes of padding are added after each 32 bit size\_\-t meaning that there are always 64 bits in a size\_\-t. This ensures that all builds of the API can operate together.\par
\par
\hypertarget{handshake_page_handshakeProcessActual}{}\subsection{Process}\label{handshake_page_handshakeProcessActual}

\begin{DoxyItemize}
\item Client attempts to connect to server via TCP.\par

\end{DoxyItemize}


\begin{DoxyItemize}
\item Client is rejected if the server is full.
\item If the client is accepted, it is sent a TCP packet from the server which contains:
\begin{DoxyItemize}
\item size\_\-t: Maximum number of clients that can be connected to server.
\item size\_\-t: Number of UDP operations (only if UDP is enabled).
\item signed char: UDP mode (only if UDP is enabled).
\item size\_\-t: Client ID of newly connected client.
\item int: Authentication code (only if UDP is enabled).
\item int: Authentication code (only if UDP is enabled).
\item int: Authentication code (only if UDP is enabled).
\item int: Authentication code (only if UDP is enabled).\par

\end{DoxyItemize}
\end{DoxyItemize}


\begin{DoxyItemize}
\item Client receives packet.
\item If UDP is disabled is the client is now fully connected and the connection process is over.
\item If UDP is enabled the process continues as follows.\par

\end{DoxyItemize}


\begin{DoxyItemize}
\item Client sends UDP packet to server. The purpose of this packet is to traverse the client's NAT and validate the UDP connection. The client repeatedly sends this packet to avoid problems with packet loss. It contains:
\begin{DoxyItemize}
\item size\_\-t: Prefix of 0 indicating the packet's purpose.
\item size\_\-t: Client's client ID.
\item int: Authentication code.
\item int: Authentication code.
\item int: Authentication code.
\item int: Authentication code.\par
\par

\end{DoxyItemize}
\end{DoxyItemize}


\begin{DoxyItemize}
\item Server receives UDP packet.
\item If the client was not validated successfully the server forcefully disconnects the client.
\item If the client is validated successfully then in order to signal that the connection process is over and tell the client that it can stop sending the UDP packet the server sends a TCP packet to the client containing no data (except for the prefix indicating a length of 0).
\end{DoxyItemize}


\begin{DoxyItemize}
\item If at any point in this process either the server's connection timeout expires or the client's timeout expires then the connection process is aborted.\par
\par

\end{DoxyItemize}\hypertarget{handshake_page_handshakeSecurity}{}\section{Security}\label{handshake_page_handshakeSecurity}
\hypertarget{handshake_page_handshakeSecurityAuthentication}{}\subsection{Authentication}\label{handshake_page_handshakeSecurityAuthentication}
Authentication codes are random integers generated by the server and are used to prevent malicious activity in the following example:
\begin{DoxyItemize}
\item Normal client begins connecting and finalizes TCP connection.
\item Server is now waiting for UDP packet which may come from a different IP or port to the TCP connection.
\item Malicious client sends UDP packet to server attempting to hijack normal client's connection.\par

\end{DoxyItemize}

In this example malicious client only needs to guess the client ID if there are no authentication codes. With authentication codes however, it is near impossible for a malicious client to hijack a connection in this way.\par
\par
\hypertarget{handshake_page_handshakeSecurityConnectionTimeout}{}\subsection{Connection Timeout}\label{handshake_page_handshakeSecurityConnectionTimeout}
If the server is spammed with connection attempts that never complete it would eventually throw an error due to running out of memory.

From the moment that a client first communicates with the server, it is allowed a set amount of time to complete the handshaking process before the process is aborted and the client is forcefully silently disconnected.\par
\par
\hypertarget{handshake_page_handshakeDisable}{}\section{Impact of disabling handshake}\label{handshake_page_handshakeDisable}
Disabling the handshake process has the following impact:
\begin{DoxyItemize}
\item All UDP commands will fail.
\item Clients will be unable to retrieve the following using in built methods/commands:
\begin{DoxyItemize}
\item Their client ID.
\item The maximum number of clients that can be connected to the server. 
\end{DoxyItemize}
\end{DoxyItemize}